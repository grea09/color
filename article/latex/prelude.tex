\usepackage{xcolor, listings}
\usepackage{latex/aaai}
\usepackage{times}
\usepackage{helvet}
\usepackage{courier}
\usepackage{algorithm}
\usepackage[noend]{algpseudocode}
\usepackage{amsthm}

% We now define the sixteen \solarized{} colors.
\definecolor{solarized-base03} {RGB}{000, 043, 054}
\definecolor{solarized-base02} {RGB}{007, 054, 066}
\definecolor{solarized-base01} {RGB}{088, 110, 117}
\definecolor{solarized-base00} {RGB}{101, 123, 131}
\definecolor{solarized-base0}  {RGB}{131, 148, 150}
\definecolor{solarized-base1}  {RGB}{147, 161, 161}
\definecolor{solarized-base2}  {RGB}{238, 232, 213}
\definecolor{solarized-base3}  {RGB}{253, 246, 227}
\definecolor{solarized-yellow} {RGB}{181, 137, 000}
\definecolor{solarized-orange} {RGB}{203, 075, 022}
\definecolor{solarized-red}    {RGB}{220, 050, 047}
\definecolor{solarized-magenta}{RGB}{211, 054, 130}
\definecolor{solarized-violet} {RGB}{108, 113, 196}
\definecolor{solarized-blue}   {RGB}{038, 139, 210}
\definecolor{solarized-cyan}   {RGB}{042, 161, 152}
\definecolor{solarized-green}  {RGB}{133, 153, 000}

\newtheorem{definition}{Definition}

%%%%%% for algorithms
\errorcontextlines\maxdimen
\algrenewcommand\alglinenumber[1]{\tiny\color{solarized-base0} #1}

% begin vertical rule patch for algorithmicx (http://tex.stackexchange.com/questions/144840/vertical-loop-block-lines-in-algorithmicx-with-noend-option)
\makeatletter
% start with some helper code
% This is the vertical rule that is inserted
    \newcommand*{\algrule}[1][\algorithmicindent]{\makebox[#1][l]{\color{solarized-base0}\hspace*{.5em}\thealgruleextra\vrule height \thealgruleheight depth \thealgruledepth}}%
% its height and depth need to be adjustable
\newcommand*{\thealgruleextra}{}
\newcommand*{\thealgruleheight}{.75\baselineskip}
\newcommand*{\thealgruledepth}{.25\baselineskip}

\newcount\ALG@printindent@tempcnta
\def\ALG@printindent{%
    \ifnum \theALG@nested>0% is there anything to print
        \ifx\ALG@text\ALG@x@notext% is this an end group without any text?
            % do nothing
        \else
            \unskip
            \addvspace{-1pt}% FUDGE to make the rules line up
            % draw a rule for each indent level
            \ALG@printindent@tempcnta=1
            \loop
                \algrule[\csname ALG@ind@\the\ALG@printindent@tempcnta\endcsname]%
                \advance \ALG@printindent@tempcnta 1
            \ifnum \ALG@printindent@tempcnta<\numexpr\theALG@nested+1\relax% can't do <=, so add one to RHS and use < instead
            \repeat
        \fi
    \fi
    }%
\usepackage{etoolbox}
% the following line injects our new indent handling code in place of the default spacing
\patchcmd{\ALG@doentity}{\noindent\hskip\ALG@tlm}{\ALG@printindent}{}{\errmessage{failed to patch}}
\makeatother

% the required height and depth are set by measuring the content to be shown
% this means that the content is processed twice
\newbox\statebox
\newcommand{\myState}[1]{%
    \setbox\statebox=\vbox{#1}%
    \edef\thealgruleheight{\dimexpr \the\ht\statebox+1pt\relax}%
    \edef\thealgruledepth{\dimexpr \the\dp\statebox+1pt\relax}%
    \ifdim\thealgruleheight<.75\baselineskip
        \def\thealgruleheight{\dimexpr .75\baselineskip+1pt\relax}%
    \fi
    \ifdim\thealgruledepth<.25\baselineskip
        \def\thealgruledepth{\dimexpr .25\baselineskip+1pt\relax}%
    \fi
    %\showboxdepth=100
    %\showboxbreadth=100
    %\showbox\statebox
    \State #1%
    %\State \usebox\statebox
    %\State \unvbox\statebox
    %reset in case the next command is not wrapped in \myState
    \def\thealgruleheight{\dimexpr .75\baselineskip+1pt\relax}%
    \def\thealgruledepth{\dimexpr .25\baselineskip+1pt\relax}%
}
% end vertical rule patch for algorithmicx


%%%% For listings (code blocks)
\lstset{
    basicstyle={\footnotesize\ttfamily},
    numbers=left,
    keywordstyle=\color{solarized-orange}\bfseries,
%    identifierstyle=\color{solarized-base02},
    stringstyle=\color{solarized-cyan},
    commentstyle=\color{solarized-base1}\itshape,
    numberstyle=\tiny\color{solarized-base0},
    columns=flexible,
    stepnumber=1,
    numbersep=5pt,
    backgroundcolor=\color{solarized-base3},
    showspaces=false,
    showstringspaces=false,
    showtabs=false,
    tabsize=2,
    captionpos=b,
    breaklines=true,
    breakatwhitespace=true,
    breakautoindent=true,
    escapeinside={\%*}{*)},
    linewidth=\textwidth,
    basewidth=0.5em,
}


%%%% For links URL
\hypersetup{colorlinks=true, linkcolor=solarized-blue, citecolor=solarized-base1, filecolor=solarized-magenta, urlcolor=solarized-cyan}

